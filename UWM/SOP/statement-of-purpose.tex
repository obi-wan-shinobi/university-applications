\documentclass{article}
\usepackage[T1]{fontenc}
\usepackage[utf8]{inputenc}
\usepackage[left=0.65in, right=0.65in, top=0.4in, bottom=0.5in]{geometry}
\usepackage{newtxmath,newtxtext}

\newcommand{\HRule}{\rule{\linewidth}{0.5mm}}
\newcommand{\Hrule}{\rule{\linewidth}{0.3mm}}

\makeatletter% since there's an at-sign (@) in the command name
\renewcommand{\@maketitle}{%
  \parindent=0pt% don't indent paragraphs in the title block
  \centering
  {\Large \bfseries\textsc{\@title}}
  \HRule\par%
  \textit{\@author \hfill \@date}
  \par
}
\makeatother% resets the meaning of the at-sign (@)

\title{Statement of Purpose}
\author{Shreyas Kalvankar}
\date{MS Applicant}

\begin{document}
  \maketitle% prints the title block
  \thispagestyle{empty}
\vspace{2pt}
\hspace{0.25in}My research interests lie in uncovering the mathematics that
drives Machine Learning and deep learning and their theoretical nature. What are
the theoretical foundations explaining the success of deep networks? Are there
inherent limits to their expressiveness? What are the potential applications of
learned representations in various domains, such as computer vision, natural
language processing, and speech recognition? How do we address the challenges of
interpretability and explainability in these learned representations? I aim to
explore these problems to develop an understanding of the theoretical limits of
Machine Learning. I am particularly interested in understanding and applying
these algorithms across engineering and natural sciences to build intelligent
systems and conduct research in physics and mathematics to solve problems such
as approximating numerical computations using neural networks, astronomical
simulations, assisted theorem proving, etc. My unwavering passion for
engineering and research is evident in my academic achievements during my
undergraduate years, reflected in my transcripts and my active involvement in
extracurricular activities. My journey has underscored the importance of further
specialization and enrichment, which I believe a master's program can provide,
laying the foundation for a future Ph.D. pursuit. In this regard, I am convinced
that the University of Wisconsin-Madison is the ideal place to nurture my
potential.
\vspace{2pt}

\hspace{0.25in}During my bachelor’s I was extremely fascinated by astrophysics
and its intersection with computer science. In my junior year, I led a team of
three and delved into the research on neural network applications in galaxy
morphology classification. Identifying galaxy morphologies has important
implications in many astronomical tasks, e.g., studying galaxy evolution. The
recent data influx in astronomy necessitates a robust and automated system for
processing large amounts of images. We aimed to set up a 7-class classification
system that classified galaxy images using CNNs, which can surpass existing
benchmarks. I was responsible for establishing a robust training pipeline and
optimizing various neural network architectures from the ground up. Our
collective efforts outperformed the second-best submission on the Kaggle Public
Leaderboard. I gained a profound understanding of the intricacies of applying
deep learning to various tasks and explored aspects such as hyperparameter
tuning, debugging, troubleshooting, and custom model design—each underscored by
a need for a profound grasp of the underlying theory. Consequently, I eagerly
anticipate working with Prof. Ilias Diakonikolas owing to his interest in the
fundamentals of deep learning, offering me an opportunity for an in-depth
exploration of the core principles of deep learning.  
\vspace{2pt}

\hspace{0.25in}The Galaxy Morphology Classification project proved to be a
stepping stone in connecting astrophysics to computer science. Soon after that
project, I began contributing to EinsteinPy, an open-source Python package
designed to address issues in General Relativity and gravitational physics. My
work specifically involved incorporating various symbolic computations, such as
the Reissner–Nordström metric and calculations for the event horizon and
ergosphere of a Kerr-Newmann black hole. While symbolic computations are
valuable in controlled environments, real-world problems often necessitate
approximate rather than exact solutions. This led me to investigate the
processes used for various physical simulations where numerical methods are
widely popular. The potential for deep learning in simulating complex system
behaviors is vast, e.g., in surrogate modeling where we can use deep learning to
approximate behaviors in complex systems allowing for faster evaluations when
original simulation using numerical methods is computationally expensive. I am
interested in leveraging AI and Machine Learning for this task in the long term.
Prof. Amos Ron’s interests in numerical analysis and the coursework he offers in
the same would be an ideal place for me to start exploring the applications of
deep learning to numerical analysis in practical engineering applications. The
coursework seems extremely fascinating and promises a comprehensive exploration
of various mathematical concepts that extend beyond my previous academic
coursework. 
\vspace{2pt}

\hspace{0.25in}For my bachelor’s thesis, I worked on utilizing Conditional GANs
for astronomical image colorization. Space archives are filled with large
amounts of low-quality, greyscale images, many of which go unnoticed. I was
interested in utilizing generative models for creating aesthetically pleasing
representations of celestial scenes. Generative models are attractive because
they enable us to better understand, create, and work with data, having many
practical and theoretical implications. In my work, I focused on tasks like
image-to-image translation and style transfer, where I realized the need for
distributional robustness in my application to maintain quality in my output
images, even in varying conditions. GANs often deal with high-dimensional data
and are used for various conditional image generation tasks; hence, it is
important to study their optimization landscape and convergence guarantees. In
my ongoing academic journey, I’m equally interested in gaining a theoretical
understanding of how various neural network models generalize, especially in
complex, high-dimensional spaces when provided with different conditions or
inputs. Undergraduate studies seldom comprehensively cover advanced machine
learning theory. In this regard, I see Prof. Yingyu Liang \& Prof. Frederic
Sala’s independent work in theoretical foundations of machine learning,
geometric machine learning \& designing ML algorithms for real-world applications
are extremely relevant to my current research interests and goals. Working with
them promises an exhilarating opportunity to immerse myself in these concepts
more formally than ever. It would give me a profound understanding and a pathway
to engaging with these fundamental theories at a higher level.
\vspace{2pt}

\hspace{0.25in}For the past two years, I have been working as an ML consultant
at Relfor Labs, an AI startup. My professional experience in Machine Learning
has further motivated me to pursue graduate studies to kick-start a research
career. The scientific landscape has been consistently evolving, but many
longstanding problems have eluded complete solutions for many years. Looking
into the future, I wish to make a difference in this significant era. My
determination to pursue a research career is unwavering, and I aspire to work in
an academic setting and eventually pursue a Ph.D. The vibrant community of
students and researchers at the University of Wisconsin-Madison is the perfect
place for me to nurture myself and further my commitment to help expedite this
scientific advancement. Thank you for considering me as a prospective student at
your university.

\end{document}
