\documentclass{article}
\usepackage[T1]{fontenc}
\usepackage[utf8]{inputenc}
\usepackage[margin=0.8in]{geometry}
\usepackage{newtxmath,newtxtext}

\newcommand{\HRule}{\rule{\linewidth}{0.5mm}}
\newcommand{\Hrule}{\rule{\linewidth}{0.3mm}}

\makeatletter% since there's an at-sign (@) in the command name
\renewcommand{\@maketitle}{%
  \parindent=0pt% don't indent paragraphs in the title block
  \centering
  {\Large \bfseries\textsc{\@title}}
  \HRule\par%
  \textit{\@author \hfill \@date}
  \par
}
\makeatother% resets the meaning of the at-sign (@)

\title{Statement of Academic Purpose}
\author{Shreyas Kalvankar}
\date{M.S. Applicant}

\begin{document}
  \maketitle% prints the title block
  \thispagestyle{empty}
My academic and professional journey in Machine Learning and Software
Development has been deeply influenced by a strong passion for creating
knowledge-based systems to tackle diverse challenges, frequently employing deep
learning algorithms. I am interested in exploring the fundamental mathematics
driving Machine Learning and deep learning.

I'm driven by questions regarding the theoretical foundations that explain deep
networks' success in approximating complex functions and the potential inherent
limits to their expressiveness. I am interested in delving into the dynamics of
training deep networks and understanding the insights gained from this process.
Moreover, I'm fascinated by the potential applications of learned
representations across domains like computer vision, natural language
processing, and speech recognition. I'm equally intrigued by the challenges
surrounding the interpretability and explainability of learned
representations.

My primary goal is to explore these theoretical quandaries and utilize this
knowledge to develop advanced learning algorithms to contribute to research in
physics and mathematics to solve problems such as approximating numerical
computations using neural networks, astronomical simulations, assisted theorem
proving, etc. and design systems capable of incremental learning and autonomous
improvement. My unwavering passion for engineering and research is evident in my
academic achievements, reflected in my transcripts and my active involvement in
extracurricular activities during my undergraduate years. My journey has
underscored the importance of further specialization and enrichment, which I
believe a master's program can provide, laying the foundation for a future Ph.D.
pursuit. In this regard, I am convinced that Carnegie Mellon University is the
ideal place to nurture my potential.

During my bachelor's I was extremely fascinated by astrophysics and its
intersection with computer science. In my junior year, I led a team of three to
apply deep learning techniques to classify galaxy morphologies. Identifying galaxy
morphologies has important implications in many astronomical tasks, e.g.,
studying galaxy evolution. The recent data influx in astronomy necessitates a
robust and automated system for processing large amounts of images. We aimed to
set up a 7-class classification system that classified galaxy images using CNNs,
which can surpass existing benchmarks. I was responsible for establishing a
robust training pipeline and optimizing various neural network architectures
from the ground up. Our collective efforts outperformed the second-best
submission on the Kaggle Public Leaderboard. This immersive journey provided a
deep understanding of intricate, deep-learning methodologies rooted in a strong
theoretical foundation. Dr. Yuanzhi Li's work in deep learning theory,
particularly her paper on \textit{'Learning Overparameterized Neural Networks via
Stochastic Gradient Descent on Structured Data,'} has been fascinating and
insightful. Her research aligns well with my interests, and having her as my
master’s advisor would be immensely valuable for me.

The Galaxy Morphology Classification project proved to be a stepping stone in
connecting astrophysics to computer science. Soon after that project, I began
contributing to \textit{EinsteinPy}, an open-source Python package designed to address
issues in General Relativity and gravitational physics. My work specifically
involved incorporating various symbolic computations, such as adding the
Reissner–Nordström metric and the calculations for the event horizon and
ergosphere of a Kerr-Newmann black hole. While symbolic computations are
valuable in controlled environments, real-world problems often necessitate
approximate rather than exact solutions. The potential for deep learning in
simulating complex system behaviors is vast. However, the challenge
remains—these problems often lack labeled data, making it difficult to apply
modern ML algorithms. Dr. Maria Balcan's research, focusing on algorithms that
learn effectively from limited data in collaboration with domain experts is
extremely relevant to address these challenges and her work resonates with my
interests and aspirations. Being mentored by her would be an incredible
opportunity.

Following my graduation, I started working as an ML engineer, later
transitioning into an ML consultant role at Relfor Labs Pvt. Ltd., an AI
startup. For the past two years, I have been responsible for building novel
neural architectures to classify audio data, represented as mel spectrograms,
and setting up the ML training pipeline. I also built a data processing pipeline
to build a proper dataset from the large database of raw audio. During this
time, I encountered challenges related to data management including data storage
and data pre-processing for training and inference, as well as problems in
deploying and maintaining ML models, which were constantly plagued by issues
like data drift and synchronization. For this very reason, The prospect of
courses like \textit{'ML with Large Datasets'} offered by Dr. Talwalkar and Dr. Gordon
excites me as it systematically covers most of the practical problems that occur
when dealing with large datasets and promises a comprehensive exploration of
various techniques for scalable deep learning and low latency inference that
extend beyond my previous academic and professional experiences.

The scientific landscape has been consistently evolving, but many longstanding
problems have eluded complete solutions for many years. Looking into the future,
I wish to make a difference in this significant era. My determination to pursue
a research career is unwavering, and I aspire to work in an academic setting and
eventually pursue a Ph.D. The vibrant community of students and researchers at
CMU is the perfect place for me to nurture myself and further my commitment to
help expedite this scientific advancement. Thank you for considering me as a
prospective student at your university.


\end{document}
