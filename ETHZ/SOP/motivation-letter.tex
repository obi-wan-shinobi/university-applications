\documentclass{article}
\usepackage[T1]{fontenc}
\usepackage[utf8]{inputenc}
\usepackage[left=0.8in, right=0.8in, top=0.5in, bottom=0.5in]{geometry}
\usepackage{newtxmath,newtxtext}

\newcommand{\HRule}{\rule{\linewidth}{0.5mm}}
\newcommand{\Hrule}{\rule{\linewidth}{0.3mm}}

\makeatletter% since there's an at-sign (@) in the command name
\renewcommand{\@maketitle}{%
  \parindent=0pt% don't indent paragraphs in the title block
  \centering
  {\Large \bfseries\textsc{\@title}}
  \HRule\par%
  \textit{\@author \hfill \@date}
  \par
}
\makeatother% resets the meaning of the at-sign (@)

\title{Motivation Letter}
\author{Shreyas Kalvankar}
\date{MSc Applicant}

\begin{document}
  \maketitle% prints the title block
  \thispagestyle{empty}
\vspace{5pt}
\hspace{0.25in}My academic and professional journey in Machine Learning and Software
Development has been shaped by a strong passion for building knowledge-based
systems to address diverse challenges, often relying on deep learning
algorithms. Through this journey, I've come to appreciate the transformative
potential of deep learning and its limitless applications in shaping our future
society. For instance, the exponential growth in astronomical data, with surveys
like Euclid projected to yield over 170 petabytes of raw input images, presents
an unprecedented opportunity for deep learning techniques to unveil valuable
insights and address once seemingly insurmountable problems. Moreover, it is
becoming increasingly feasible to envision interpretability in deep learning
systems that can accurately model complex problems like galaxy formation,
furthering research and innovation and potentially leading to new deep learning
algorithms. Looking ahead, I aspire to be at the forefront of these fields and
contribute to solutions that benefit humanity. My journey has underscored the
importance of further specialization and enrichment, which I believe a master's
program can provide, laying the foundation for a future Ph.D. pursuit. In this
regard, I am convinced that ETH Zürich, renowned for its world-class education,
is the ideal place to nurture my potential.
\vspace{5pt}

\hspace{0.25in}My unwavering passion for engineering and research is evident in my academic
achievements, reflected in my transcripts and my active involvement in
extracurricular activities during my undergraduate years. This passion was
further cemented in my junior year as I delved into the research on neural
network applications in galaxy morphology classification. Identifying galaxy
morphologies has important implications in many astronomical tasks, e.g.,
studying galaxy evolution. The data influx in astronomy necessitates a robust
and automated system. We aimed to set up a 7-class classification system that
classified galaxy images using CNNs, which can surpass existing benchmarks. I
was responsible for establishing a robust training pipeline and optimizing
various neural network architectures from the ground up. Our collective efforts
outperformed the second-best submission on the Kaggle Public Leaderboard. I
gained a profound understanding of the intricacies of applying deep learning to
tasks, exploring aspects such as hyperparameter tuning, debugging,
troubleshooting, and custom model design—each underscored by a need for a
profound grasp of the underlying theory. Consequently, I eagerly anticipate
coursework such as 'Deep Learning' by Prof. Dr. Thomas Hofmann, Prof. Dr.
Fernando Cruz, and Prof. Dr. Nathanaël Perraudin, offering a more in-depth
exploration of the core principles of deep learning and rigorous mathematics
compared to my prior senior-year coursework. I'm also drawn to the intriguing
work of Prof. Hofmann and his colleagues at the Data Analytics lab, where they
focus on interpretability and theoretical understanding of deep learning. Dr.
Hofmann's expertise makes him a compelling choice as a Master's tutor. I'm
equally excited about collaborating with Dr. Fluri as his deep learning
expertise in astrophysics and related areas aligns seamlessly with my past work
and future interests.
\vspace{5pt}

\hspace{0.25in}For my bachelor thesis, I worked on utilizing Conditional GANs
for astronomical image colorization. Space archives are filled with large
amounts of low quality, greyscale images, many of which go unnoticed. I was
interested in utilizing generative models for creating aesthetically pleasing
representations of celestial scenes. Generative models are attractive because
they enable us to better understand, create, and work with data. Besides the
artistic implications of my work, studying these models was insightful because
of their complexity and innovative training methodology. In my work, I focused
on tasks like image-to-image translation and style transfer, where I realized
the need for distributional robustness in my application to maintain quality in
my output images even when subjected to varying conditions. GANs often deal with
high-dimensional data and are used for various conditional image generation
tasks. In my ongoing academic journey, I’m equally enthralled in gaining a
theoretical understanding of how various neural network models generalize,
especially in complex, high-dimensional spaces. This could help answer questions
about the generalization properties of such models when provided with different
conditions or inputs. In this regard, the ongoing research led by Prof. Dr.
Fanny Yang and the Statistical Machine Learning group stands out as
exceptionally captivating.  Undergraduate studies seldom comprehensively cover
advanced machine learning theory. Therefore, I see the coursework in 'Guarantees
for Machine Learning' as an exhilarating opportunity to immerse myself in these
concepts more formally than ever. It promises a more profound understanding and
a pathway to engaging with these fundamental theories at a higher level.
\vspace{5pt}

\hspace{0.25in}Following my graduation, I started as an ML engineer, later transitioning into
an ML consultant role at Relfor Labs Pvt. Ltd., an AI startup, where I
encountered challenges related to data management, particularly in storing and
processing large amounts of raw audio data for training. I also realized certain
problems, like data drift and model synchronization, which occur while
maintaining \& deploying ML models. Even in my current role as a Software
Developer at Dalton Maag Ltd, I've grappled with data-related challenges,
particularly in processing \& integrating copious amounts of disparate data into
our systems. The flexibility of the master’s program allows me to select a minor
in Data Management Systems, which will help me address these challenges.
Additionally, the prospect of courses like ‘Big Data’ excites me as they promise
a comprehensive exploration of various data processing techniques that extend
beyond my previous academic and professional experiences.
\vspace{5pt}

\hspace{0.25in}The scientific landscape has been consistently evolving, but many longstanding
problems have eluded complete solutions for many years. Looking into the future,
I wish to make a difference in this significant era by contributing to this
field. The vibrant community of students and researchers at ETH Zurich is the
perfect place for me to nurture myself and further my commitment to help
expedite this scientific advancement. 


\end{document}
