\documentclass{article}
\usepackage[T1]{fontenc}
\usepackage[utf8]{inputenc}
\usepackage[margin=1in]{geometry}
\usepackage{newtxmath,newtxtext}
\usepackage[authoryear]{natbib}
\usepackage{xcolor}
\usepackage[colorlinks=true,linkcolor=blue,citecolor=blue,urlcolor=blue]{hyperref}
\usepackage{titlesec}

\newcommand{\HRule}{\rule{\linewidth}{0.5mm}}
\newcommand{\Hrule}{\rule{\linewidth}{0.3mm}}

\renewcommand{\refname}{Publications}

\setcitestyle{authoryear,round,semicolon,maxcitenames=2}

\titleformat{\section}{\normalfont\normalsize\bfseries}{\thesection}{1em}{}[\vspace{-\parskip}]

\makeatletter% since there's an at-sign (@) in the command name
\renewcommand{\@maketitle}{%
  \parindent=0pt% don't indent paragraphs in the title block
  \centering
  {\Large \bfseries\textsc{\@title}}
  \HRule\par%
  \textit{\@author \hfill \@date}
  \par
}
\makeatother% resets the meaning of the at-sign (@)

\setlength{\parindent}{0.25in}
\setlength{\parskip}{5pt}

\title{Academic Statement of Purpose}
\author{Shreyas Kalvankar}
\date{M.S. Applicant}

\begin{document}
  \maketitle% prints the title block
  \thispagestyle{empty}

\section*{Long Term Degree Objectives}

\hspace*{0.25in}My professional and undergraduate research experiences have motivated me to
pursue graduate studies to kick-start a research career. Many longstanding
problems in computer science and machine learning have eluded complete solutions
for many years, e.g. the theory explaining the success of deep neural networks.
Looking into the future, I wish to make a difference in this significant era. My
determination to pursue a research career is unwavering, and I aspire to work in
an academic setting and eventually pursue a Ph.D. The vibrant community of
students and researchers at Purdue University is the perfect place for me to
nurture myself and further my commitment to help expedite this scientific
advancement.

\section*{Research}

\hspace*{0.25in}During my bachelor's I was extremely fascinated by astrophysics
and its intersection with computer science. In my junior year, I led a team of
three and delved into the research on neural network applications in galaxy
morphology classification. Identifying galaxy morphologies has important
implications in many astronomical tasks, e.g., studying galaxy evolution. The
recent data influx in astronomy necessitates a robust and automated system for
processing large amounts of images. We aimed to set up a 7-class classification
system that classified galaxy images using CNNs, which can surpass existing
benchmarks. I was responsible for establishing a robust training pipeline and
optimizing various neural network architectures from the ground up. Our
collective efforts outperformed the second-best submission on the Kaggle Public
Leaderboard. I gained a profound understanding of the intricacies of applying
deep learning to various tasks and explored aspects such as hyperparameter
tuning, debugging, troubleshooting, and custom model design. \citep{kalvankar2020galaxy}

For my bachelor's thesis, I worked on utilizing Conditional GANs for
astronomical image colorization. Space archives are filled with large amounts of
low-quality, greyscale images, many of which go unnoticed. I was interested in
utilizing generative models for creating aesthetically pleasing representations
of celestial scenes. Generative models are attractive because they enable us to
better understand, create, and work with data, having many practical and
theoretical implications. Exploring GANs was a natural progression to enhance my
skills after working extensively with CNN architectures. GANs introduce a new
layer of complexity with their unique architecture and loss functions, providing
an ideal opportunity to deepen my understanding of developing intricate deep
learning systems. \citep{kalvankar2022astronomical}

In addition to helping hone my research skills and nurture my critical thinking
abilities, my role as the project lead in these projects was a significant
learning experience that taught me various aspects of management, communication,
and project planning.

The Galaxy Morphology Classification project proved to be a stepping stone in
connecting astrophysics to computer science. Soon after that project, I began
contributing to EinsteinPy, an open-source Python package designed to address
issues in General Relativity and gravitational physics. My work specifically
involved incorporating various symbolic computations, such as the
Reissner–Nordström metric and calculations for the event horizon and ergosphere
of a Kerr-Newmann black hole. I learned crucial concepts in physics and general
relativity and the experience highlighted potential applications of deep
learning within this domain. \citep{bapat2020einsteinpy}

\section*{Future Research Interests}

\hspace*{0.25in}My research interests lie in uncovering the mathematics that drives Machine
Learning and deep learning and their theoretical nature. What are the
theoretical foundations explaining the success of deep networks and are there
inherent limits to their expressiveness? What are the potential applications of
learned representations in various domains, such as computer vision, natural
language processing, and speech recognition? How do we address the challenges of
interpretability and explainability in these learned representations? I aim to
explore these problems to develop an understanding of the theoretical limits of
Machine Learning. I find Prof. Rajiv Khanna's work in interpretability and
generalization guarantees fascinating in this regard and would like to work with
him to explore these problems.

I am particularly interested in understanding and applying these algorithms
across engineering and natural sciences to build intelligent systems and conduct
research in physics and mathematics to solve problems such as approximating
numerical computations using neural networks, astronomical simulations, assisted
theorem proving, etc. Prof. Yexiang Xue's interests in AI for scientific
discovery align extremely well with mine, and I wish to work with him in this
domain. A particularly interesting problem that I wish to explore is applying
deep learning to gravitational simulations. In simulations, gravitational
interactions are currently computed using hierarchical tree algorithms like the
Barnes-Hut \& Fast Multipole Method. One possible approach is to model the
gravitational interactions of various particles as nodes of a graph and learn
the patterns as they evolve over time by using Graph Neural Networks. I believe
Prof. Bruno Ribeiro's expertise in Graph Neural Networks can help me immensely
in my future research pursuits. 

\section*{Other Comments}

\hspace*{0.25in}Because of my interest in simulations, I look forward to courses in "Numerical
Computing" as they promise a comprehensive exploration of various mathematical
concepts that extend beyond my previous academic coursework.

My research experiences in deep learning have underscored a need for a profound
grasp of the underlying theory. Consequently, I eagerly anticipate the advanced
topics in coursework such as "Foundations Of Deep Learning" (CS 58700), offering
a more in-depth exploration of the core principles of deep learning and rigorous
mathematics compared to my prior senior-year coursework.

In my ongoing academic journey, I’m equally interested in gaining a theoretical
understanding of how various neural network models, like GANs, generalize,
especially in complex, high-dimensional spaces when provided with different
conditions or inputs. Undergraduate studies seldom comprehensively cover
advanced machine learning theory. In this regard, I see Prof. Steve Hanneke’s
coursework in "Machine Learning Theory" as an exhilarating opportunity to
immerse myself in these concepts more formally than ever. It promises a more
profound understanding and a pathway to engaging with these fundamental theories
at a higher level.

I believe my research interests, background, and career goals align extremely
well the offerings of Purdue university. Having set ambitious goals in the
intersection of deep learning and astrophysics, my background in physics,
coupled with a robust understanding of crucial concepts such as general
relativity, positions me at the nexus of theoretical and applied research. The
institution's emphasis on pushing the boundaries of knowledge and providing a
supportive academic environment resonates with my commitment to contributing
meaningfully to the scientific community. Thank you for considering me as a
prospective student at your university.

\bibliographystyle{abbrvnat}
\bibliography{bibliography}
\end{document}

