\documentclass[12pt]{article}
\usepackage[T1]{fontenc}
\usepackage[utf8]{inputenc}
\usepackage[margin=0.75in]{geometry}
\usepackage{newtxmath,newtxtext}

\newcommand{\HRule}{\rule{\linewidth}{0.5mm}}
\newcommand{\Hrule}{\rule{\linewidth}{0.3mm}}

\makeatletter% since there's an at-sign (@) in the command name
\renewcommand{\@maketitle}{%
  \parindent=0pt% don't indent paragraphs in the title block
  \centering
  {\Large \bfseries\textsc{\@title}}
  \HRule\par%
  \textit{\@author \hfill \@date}
  \par
}
\makeatother% resets the meaning of the at-sign (@)

\title{Personal Essay}
\author{Shreyas Kalvankar}
\date{M.S. Applicant}

\begin{document}
  \maketitle% prints the title block
  \thispagestyle{empty}

\vspace{5pt}

Since childhood, my curiosity about science has sparked an
interest in many fields, including computer science and mathematics. As a kid, I
constantly strived to understand how complicated systems work, which fueled my
desire to build such systems in the future. The idea of complex patterns arising
from interactions of simple components stems from the notion of emergence
mentioned in Hofstadter’s \textit{"Gödel, Escher, Bach."} This idea resonated strongly
with me and was further reinforced in my mind as I started researching deep
learning systems for galaxy morphology classification. These systems elegantly
constructed the fundamental building blocks found in linear algebra, resulting
in straightforward yet intuitive descriptions of complex behaviours of pattern
recognition. This project proved to be a turning point in my life and helped me
discover my primary area of interest.

My academic interests have enabled me to explore novel approaches for uniting
two seemingly unconnected fields. The skills I developed working on applying
deep learning to galaxy morphology classification paved the way for further
investigation into connecting the fields of astrophysics and computer science.
Throughout my college, I have embraced several roles -- serving as an academic
mentor for the theory of computation, machine learning courses, leading teams in
both academic and cocurricular domains, including the robotics club and my
bachelor’s thesis project. I have taken numerous lectures talking about the
potential of computer science in mathematics and physics. My teaching
experiences and research endeavours have instilled in me a desire to contribute
to the academic community. I am excited to use my teaching and mentoring skills
by joining the \textit{Purdue Compuer Science Graduate Student Association} at Purdue
University.

Apart from academics, I am interested in linguistic anthropology and spend my
free time studying languages like Latin and Gaelic and their origins in
geographical and cultural contexts. I am also interested in Japanese and have
been learning \text{shodō}, a form of Japanese calligraphy. Drawing a letter is a
daunting task because of the intricacies involved but it is extremely
gratifying. As I started learning Japanese and calligraphy, I was drawn to the
world of typography and soon after graduation, I started working at Dalton Maag,
a type design studio in London. My professional experiences proved to be quite
different than my academic ones. I found myself navigating through a landscape
of problems which extended far beyond my academic curriculum. Drawing Japanese
letters, especially Kanji, is a time-consuming task. To address this, I
developed a system to generate these glyphs using a set of design parameters
like stroke length, width, brush pressure, etc. To provide a consistent design,
I used genetic algorithms to fine-tune these parameters by learning from a set
of input glyphs. I realized how my past research experiences had helped me
discover new ways of thinking and apply my skills effectively to certain
problems. Working at Dalton Maag also provided me the opportunity to collaborate
with a diverse and global team. Interacting with colleagues from various corners
of the world allowed me to delve into their cultures, fostering an appreciation
for linguistic diversity. This ignited a curiosity about the connections between
languages and scripts, sparking my interest in linguistic anthropology. I spend
my free time studying languages like Latin and Gaelic and their origins in
geographical and cultural contexts. I wish to find individuals who share a
similar affinity for studying languages and typography to create a
\textit{Linguistic Artistry} club at Purdue. 

Reflecting on my journey in art and science, I find a striking resemblance to
the exploration of the interconnectedness of math, music, and art in
Hofstatder’s \textit{Eternal Golden Braid}. Much like the dialogues in the book between
Achilles and the tortoise that unravel the interconnected nature of mathematics,
art, and music, my own experiences seem to echo the profound interplay between
these seemingly distinct realms. My academic and professional experiences have
provided me with an eclectic background and a diverse set of interests. I find
that this diversity resonates well with the vibrant community at Purdue and I am
confident that I can bring valuable contributions to further enrich this dynamic
tapestry.

\end{document}
