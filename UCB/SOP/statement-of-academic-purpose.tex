\documentclass[11pt]{article}
\usepackage[T1]{fontenc}
\usepackage[utf8]{inputenc}
\usepackage[left=1in, right=1in, top=0.5in, bottom=0.5in]{geometry}


\newcommand{\HRule}{\rule{\linewidth}{0.5mm}}
\newcommand{\Hrule}{\rule{\linewidth}{0.3mm}}

\makeatletter% since there's an at-sign (@) in the command name
\setlength{\parskip}{5pt}
\renewcommand{\@maketitle}{%
  \parindent=0pt% don't indent paragraphs in the title block
  \centering
  {\Large \bfseries\textsc{\@title}}
  \HRule\par%
  \textit{\@author \hfill \@date}
  \par
}
\makeatother% resets the meaning of the at-sign (@)

\title{Statement of Purpose}
\author{Shreyas Kalvankar}
\date{Master of Information Management and Systems applicant}

\begin{document}
  \maketitle% prints the title block
  \thispagestyle{empty}
  \vspace{5pt}

\hspace{0.25in}My research interests lie in uncovering the mathematics that drives Machine
Learning and deep learning and their theoretical nature. What are the
theoretical foundations explaining the success of deep networks in approximating
complex functions, and are there inherent limits to their expressiveness? What
insights can we gain into the dynamics of training deep networks? What are the
potential applications of learned representations in various domains, such as
computer vision, natural language processing, and speech recognition? How do we
address the challenges of interpretability and explainability in learned
representations? I aim to explore these problems and develop an understanding of
the fundamental theoretical limits of Machine Learning. With this goal in mind,
I want to pursue my Master’s in Information Systems Management at UC Berkeley. I
am particularly interested in effectively applying these algorithms to build
intelligent systems to conduct research in physics and mathematics to solve
problems such as approximating numerical computations using neural networks,
exoplanet discovery, assisted theorem proving, etc.

\hspace{0.25in}My undergraduate research endeavors have been in applying deep learning
techniques to astronomy and astrophysics, a domain that is undergoing
significant transformations brought about by exponential growth in our abilities
to collect and process data. I attempted to solve this problem using deep
learning algorithms and, along with my colleagues, worked on a project to apply
CNNs for Galaxy Morphology Classification, which subsequently turned into my
first authored research paper. Galaxy morphologies are studied by astronomers to
study dark matter distribution, galaxy evolution, and galaxy interactions. The
problem involved correctly classifying galaxy images into various morphologies
and also predicting the presence of certain features like rings, spirals, etc.
We sought to create a 7-class system encompassing most galaxy morphologies. I
set up the entire training pipeline and the model architectures, enabling robust
training of 7 EfficientNet models. Our results were better than those at the
second position on the Kaggle Public Leaderboard, effectively making it the
second-best submission. This project made me realize the impact intelligent
systems can have in streamlining scientific research. 

 \hspace{0.25in}I continued exploring more applications of such systems to astronomy and started
working with Dr. Snehal Kamalapur in the Intelligent Systems Lab. I looked into
developing an efficient process to colorize and up-scale unprocessed
astronomical images that lie dormant and unseen in extensive space archives. I
was fortunate to present this study at the \textit{Informatik 2022} conference’s Astro ML
workshop in Hamburg. My role as the project lead was a significant learning
experience that taught me various aspects of management, communication, and
project planning. Through this project, I honed my research skills, nurtured my
critical thinking abilities, and sharpened my decision-making capabilities. 

 \hspace{0.25in}My interest in this field expanded my research thinking beyond the confines of
ML. I started contributing to an open-source project called EinsteinPy.
EinsteinPy is an open-source pure Python package that studies problems arising
in General Relativity and gravitational physics. My work specifically involved
incorporating various symbolic computations, such as the Reissner–Nordström
metric and calculations for the event horizon and ergosphere of a Kerr-Newmann
black hole. Working on problems motivated by the challenges in science and
engineering requires understanding other fields like numerical methods,
computational geometry, etc. I am particularly interested in pursuing this area
of research at the Berkeley AI Research Lab. Prof. Dr. Krishnapriyan’s interests
in developing physics-inspired machine learning methods seem to align perfectly
with mine, and I would like to explore more such research at the BAIR Lab.

 \hspace{0.25in}After completing my degree, I embarked on my professional journey, starting as a
Machine Learning Engineer and later transitioning to a Machine Learning
Consultant role at the startup Relfor Labs. My primary focus revolved around
applying deep learning to audio data analysis and classification by devising
novel architectures. I worked on designing a system for pre-processing large
amounts of raw audio data by converting it into mel spectrograms, refining the
labels. I set up an end-to-end training pipeline to streamline experimentation.
Throughout my work on this and various other ML and data-intensive projects, I
frequently encountered the challenge of managing and comprehending extensive
datasets. I have experienced firsthand the complexities of automating an
end-to-end ML pipeline which can be sustainable in the long term, including data
quality and consistency issues, model drift, version control, and debugging. My
experience in developing ML models and designing ML pipelines, along with my
challenges, has motivated me to tackle these issues by creating tools that
streamline data management and ML processes. I have followed the work of Prof.
Dr. Parameswaran and his student Ms. Shankar and I want to contribute to their
efforts of simplifying data science and building tools to manage ML pipelines
effectively. 

 \hspace{0.25in}I've been a full-time Software Developer at Dalton Maag, a type-design studio in
London, for the past two years. Shortly after joining, I became involved in a
project to design a proof-of-concept system that utilizes genetic algorithms to
generate CJK glyphs automatically. Because of the sheer volume of these glyphs;
often ranging in thousands, designing and drawing them is an expensive,
time-consuming task. I was responsible for creating a framework that could learn
from a few hundred glyphs and draw the remaining ones in the same style. This
experience highlighted the incredible potential of harnessing data to create
intelligent systems across various fields, profoundly influencing my interests.

 \hspace{0.25in}My exposure to the industry has ignited a strong passion for crafting
comprehensive systems tailored to meet the specific needs of businesses,
particularly those reliant on effective data and information utilization. One
such system that I worked on, Pricebot, stands out as a prime example.  I was
responsible for developing and integrating the pricing model which is used to
create project quotes and plans by understanding the business needs. This tool
completely automates a previously manual process and consistently delivers
precise results which effectively prevents project overruns and unforeseen
expenses, cutting down planning time significantly and saving valuable
resources.

 \hspace{0.25in}My goal is to leverage the mathematical foundations of machine learning to apply
AI to scientific applications, developing tools that streamline data management
and systems for automating business processes through information systems. The
MIMS program at the School of Information is a unique blend of data science,
social good and technology. The interdisciplinary nature of the program makes it
possible to marry multiple domains together and use information to build
technology that has real impact. Conversations with an I-School alumnus,
Rajvardhan Oak, have convinced me that the MIMS program would prepare me well
for research careers in the industry, or further doctoral studies. My
determination to pursue a research career is unwavering, and I aspire to work in
an academic setting and eventually pursue a Ph.D. Thank you for considering me
as a prospective student at your university.

\end{document}
